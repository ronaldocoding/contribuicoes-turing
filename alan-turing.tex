\documentclass[
  % -- opções da classe memoir --
  12pt,     % tamaho da fonte
  openright,      % capítulos começam em págína ímpar
  oneside,      % para impressão em recto e verso. Oposto a oneside
  a4paper     % tamanho do papel        % o último idioma é o principal do documento
  ]{abntex2}

% ---
% Pacotes básicos
% ---
\usepackage{lmodern}			% Usa a latin modern
\usepackage[T1]{fontenc}		% Selecao de codigos de fonte.
\usepackage[utf8]{inputenc}		% Codificacao do documento (conversão automática dos acentos)
\usepackage{indentfirst}		% Indenta o primeiro parágrafo de cada seção.
\usepackage{color}				% Controle das cores
\usepackage{microtype} 			% para melhorias de justificação
% ---

% ---
% Pacotes de citações
% ---
\usepackage[brazilian,hyperpageref]{backref}	 % Paginas com as citações na bibliografia
\usepackage[alf]{abntex2cite}	% Citações padrão ABNT
% ---

% ---
% Configurações do pacote backref
% Usado sem a opção hyperpageref de backref
\renewcommand{\backrefpagesname}{Citado na(s) página(s):~}
% Texto padrão antes do número das páginas
\renewcommand{\backref}{}
% Define os textos da citação
\renewcommand*{\backrefalt}[4]{
	\ifcase #1 %
		Nenhuma citação no texto.%
	\or
		Citado na página #2.%
	\else
		Citado #1 vezes nas páginas #2.%
	\fi}%
% ---

\renewcommand\bibsection{\section*{\bibname}}

% ---
% Informações de dados para CAPA e FOLHA DE ROSTO
% ---
\titulo{Contribuições de Alan Turing à Ciência da Computação}
\autor{Ronaldo Costa de Freitas}
\local{Manaus, Amazonas}
\data{2021}
\orientador{prof. Ricardo da Silva Barbosa}
\instituicao{
  Universidade do Estado do Amazonas -- UEA
  \par
  Faculdade de Sistemas de Informação}
% O preambulo deve conter o tipo do trabalho, o objetivo,
% o nome da instituição e a área de concentração
\preambulo{Trabalho acadêmico apresentado à Faculdade de Sistemas de Informação da Universidade do Estado do Amazonas como requisito para obtenção de nota parcial à matéria de Introdução à Computação ministrada pelo prof. Ricardo da Silva Barbosa.}
% ---

% ---
% Configurações de aparência do PDF final

% alterando o aspecto da cor azul
\definecolor{blue}{RGB}{41,5,195}

% informações do PDF
\makeatletter
\hypersetup{
     	%pagebackref=true,
		pdftitle={\@title},
		pdfauthor={\@author},
    	pdfsubject={\imprimirpreambulo},
	    pdfcreator={LaTeX with abnTeX2},
		pdfkeywords={abnt}{latex}{abntex}{abntex2}{trabalho acadêmico},
		colorlinks=true,       		% false: boxed links; true: colored links
    	linkcolor=blue,          	% color of internal links
    	citecolor=blue,        		% color of links to bibliography
    	filecolor=magenta,      		% color of file links
		urlcolor=blue,
		bookmarksdepth=4
}
\makeatother
% ---

% ---
% Espaçamentos entre linhas e parágrafos
% ---

% O tamanho do parágrafo é dado por:
\setlength{\parindent}{1.3cm}

% Controle do espaçamento entre um parágrafo e outro:
\setlength{\parskip}{0.2cm}  % tente também \onelineskip

% ---
% compila o indice
% ---
\makeindex
% ---

% ----
% Início do documento
% ----
\begin{document}

% Seleciona o idioma do documento (conforme pacotes do babel)
%\selectlanguage{english}
\selectlanguage{brazil}

% Retira espaço extra obsoleto entre as frases.
\frenchspacing

% ----------------------------------------------------------
% ELEMENTOS PRÉ-TEXTUAIS
% ----------------------------------------------------------
%\pretextual

% ---
% Capa
% ---
\imprimircapa
% ---

% ---
% Folha de rosto
% ---
\imprimirfolhaderosto
% ---

% ---
% inserir o sumario
% ---
\pdfbookmark[0]{\contentsname}{toc}
\tableofcontents*
\cleardoublepage
% ---

% ----------------------------------------------------------
% ELEMENTOS TEXTUAIS
% ----------------------------------------------------------
\textual

% ------------------------------------------------------------
% Texto principal vai aqui
% ------------------------------------------------------------
\chapter*{}
\section*{Contribuições de Alan Turing à Ciência da Computação}
\addcontentsline{toc}{section}{Contribuições de Alan Turing à Ciência da Computação}
% ------------------------------------------------------------

Alan Mathison Turing foi um matemático inglês pioneiro da ciência da computação teórica e inteligência artificial. Ele nasceu em 23 de junho de 1912 em Londres e morreu em 7 de junho de 1954, vítima de envenenamento por cianeto, em sua casa em Cheshire \cite{Hassall2014}.  Na época um inquérito concluiu que foi suicídio, embora isso tenha sido contestado mais recentemente pelo professor Jack Copeland, um especialista na vida do cientista, que atribuiu sua morte à inalação acidental de fumaça de cianeto durante um experimento. Dois anos antes ele fora levado a julgamento e declarado culpado por "ato de indecência grosseira", ele era homossexual. Sua punição foi sofrer castração química \cite{Library2019}.

Durante o tempo em que estudou na Universidade de Princeton, Alan desenvolveu a noção de uma "máquina de computação universal". Essa máquina seria capaz de escanear, ou ler, instruções codificadas em uma fita de comprimento infinito. À medida que o escâner se movesse de um quadrado da fita para o outro, respondendo aos comandos sequenciais e modificando sua resposta mecânica se assim ordenado, o resultado desse processo poderia replicar o pensamento lógico humano. Ao trocar as instruções contidas na fita, poderia se fazer a máquina desempenhar as funções de todas as categorias de máquinas. Em outras palavras, a \textit{Universal Turing Machine}, como hoje é conhecida, poderia tanto calcular números quanto jogar xadrez ou fazer qualquer outra coisa de natureza semelhante \cite{Gray1999}.

Além de ajudar a fundamentar o conhecimento do que seria o computador digital eletrônico, Turing também criou, entre os anos de 1940 e 1943: (1) uma máquina chamada \textit{Bombe} capaz de decifrar as mensagens codificadas pelo \textit{Enigma}, dispositivo usado pelas forças armadas da Alemanha para se comunicarem de forma segura durante a Segunda Guerra Mundial, (2) uma técnica complexa de quebra de código nomeada \textit{Turingery}, usada contra as mensagens com cifra de Lorenz produzida pela nova máquina de codificação dos alemães e (3) um sistema de fala chamado \textit{Delilah}, em que era possível codificar e decodificar comunicações por voz \cite{Gray1999}.

Após encurtar a duração da Segunda Guerra Mundial em, no mínimo, 2 anos, salvando incontáveis vidas, o matemático tornou-se subdiretor do Laboratório de Computação da Universidade de Manchester em 1949 e escreveu seu famoso artigo \textit{Computing Machinery and Intelligence} em 1950. Nesse artigo, ele abordou pela primeira vez o assunto da Inteligência Artificial (IA) e concebeu um método hoje chamado Teste de Turing. Segundo ele, esse método conseguiria determinar se o comportamento demonstrado por uma máquina poderia ser considerado "inteligente".  O teste influenciou significativamente pesquisas na área de IA. As contribuições de Alan Turing à Ciência da Computação foram inúmeras e hoje são reconhecidas e comemoradas durante a realização anual do "Prêmio Turing", considerado o "Prêmio Nobel da computação" \cite{Gray1999}.

% ----------------------------------------------
% Referências aqui
% ----------------------------------------------
\chapter*{}
\addcontentsline{toc}{section}{Referências}
\bibliography{referencias.bib}

\end{document}
